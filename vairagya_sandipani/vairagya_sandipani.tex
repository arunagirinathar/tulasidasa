\def\DevnagVersion{2.15}\documentclass{book}
\usepackage[a5paper]{geometry}
\usepackage{hyperref}
\usepackage[all]{nowidow}
\setnowidow
\setnoclub

\newcommand{\flush}{\hspace*{\fill}}
\newcommand{\pa}{\hspace*{.9cm}\\*}
\usepackage{type1cm}
\newcommand{\trans}{\parindent=0pt\let\thefootnote\relax\footnotetext}
\newcommand{\tmark}{\let\thefootnote\relax\footnotemark}


\usepackage{afterpage}

\newcommand\blankpage{%
    \null
    \thispagestyle{empty}%
    \addtocounter{page}{-1}%
    \newpage}
    
\usepackage{titlesec}
\titleformat{\section}{}{}{0pt}{\normalsize\scshape\centering}
\titlespacing{\section}{0cm}{0cm}{.2cm}
\setcounter{secnumdepth}{0} % sections are level 1


\usepackage{devanagari}
\title{\\*[0cm]{\dn\Huge v\4rA`ys\306wdFpnF}\\ \rn{\scshape Vair\=agyasand\={\i}pan\={\i}\\ {\large\itshape A Renunciation Kindler}}}
\author{\emph{Composed by}\\{\dn\large \399wFm\38Bwo-vAEm\7{t}lsFdAsvy\0,}\\ \rn{\scshape Srimad Goswami Tulasidas}\\[1cm]\textit{Translated into English by}\\{\scshape Ajai Kumar Chhawchharia}\\*[2cm]}
\date{\hline\hline{\dn\large jys\2v(srFyA mAg\0Efr\8{p}EZ\0mA}\\\today}
\setlength\parindent{0pt}
\setlength\parskip{12pt}
\newcommand{\secm}{\\*[1cm]}

\begin{document}
\thispagestyle{empty}
\begin{center}
{\dn\dnpen .. \399wFgZ\?ffArdA\7{g}z<yo nm, ..}\\[.5cm]
\end{center}
\vspace*{.5cm}
\begin{center}
{\dn\Huge\Huge v\4rA`ys\306wdFpnF}\\[0.5cm]{\Huge\scshape Vair\=agyasand\={\i}pan\={\i}}\\[.5cm]{\Large\itshape A Renunciation Kindler}\\[2cm]

{\itshape\Large Composed by\\ {\bfseries\'Sr\={\i}mad-Gosv\=am\={\i} Tulas\={\i}d\=asa-ji}\\[0.3cm]{\huge \&}\\translated into English by\\[0.3cm] \emph{Ajai Kumar Chhawchharia}}\\*[1cm]


{\itshape\large Edited \& Revised by\\[0.1cm]}
{\scshape\large ``\'Sivap\=adah\d{r}daya\.m''}\\[1cm]

---2014---

\end{center}
%\newpage\clearpage\thispagestyle{empty}
\clearpage\thispagestyle{empty}\mbox{}\clearpage



\newpage
\thispagestyle{empty}

\section*{Introduction}
{\large \emph{Vair\=agya Sand\={\i}pan\={\i}} is a poetically, philosophically and intellectually mature work of \emph{\'Sr\={\i}mad Gosw\=am\={i} Tulas\={\i}d\=asa}--the saint poet of North India having no parallel in the world. It was perhaps one of the last chapters of his life and works. As the very name indicates, it kindles or promotes---atleast indentes to do so---dispassion and devotion. It is on the pattern of \emph{Tirukkural} of \emph{Tiruvalluvar} or \emph{Vair\=agya \'Satakam} or \emph{Bhart\d{r}hari}. A small booklet containing a few \emph{Dohas} and \emph{Chaupais}, approximately fifty, so to say, but the ideas contained therein, are grand and irrefutable as he lived and breathed in dispassion from the world and strong attachment to his \emph{R\=ama}. I expect that the readers will find it a very good guide and a guard against the undue worldly allurements. It is not only worth reading but translating too into life.
\begin{flushright}
---The Editor\\*[0.1cm]
Kalyana Kalpataru.
\end{flushright}

\textbf{N.B.} \textit{The English translation is a reproduction of the translation that appeared in \emph{Kalyana-Kalpataru} from February through May 2004. The translation was penned by \emph{Ajai Kumar Chhawchharia}
}

\newpage
\thispagestyle{empty}
\section*{System of transliteration}
\begin{center}
{\Large Vowels}

\begin{tabular}{c c c c c c c c}

{\dn\Large a} & {\dn\Large aA} & {\dn\Large i} & {\dn\Large I} & {\dn\Large u} & {\dn\Large U} & {\dn\Large \31Bw} & {\dn\Large \311w} \\
a & \=a & i & \={\i} & u & \=u & \d{r} & \d{\={r}} \\*[.45cm]


{\dn\Large \318w} & {\dn\Large \319w} & {\dn\Large e} & {\dn\Large e\?} & {\dn\Large ao} & {\dn\Large aO} & {\dn\Large \=\2} & {\dn\Large ,} \\
\d{l} & \d{\=l} & e & ai & o & au & \.m & \d{h}\\*[0.25cm]
\end{tabular}

{\Large Consonants}


\begin{tabular}{c c c c c}
{\dn\Large k} & {\dn\Large K} & {\dn\Large g} & {\dn\Large G} & {\dn\Large R} \\
ka & kha & ga & gha & \.na\\*[0.25cm]

{\dn\Large c} & {\dn\Large C} & {\dn\Large j} & {\dn\Large J} & {\dn\Large \31Aw} \\
ca & cha & ja & jha & \~na\\*[0.25cm]

{\dn\Large V} & {\dn\Large W} & {\dn\Large X} & {\dn\Large Y} & {\dn\Large Z} \\

\d{t}a & \d{t}ha & \d{d}a & \d{d}ha & \d{n}a \\*[0.25cm]
{\dn\Large t} & {\dn\Large T} & {\dn\Large d} & {\dn\Large D} & {\dn\Large n} \\
ta & tha & da & dha & na\\*[0.25cm]

{\dn\Large p} & {\dn\Large P} & {\dn\Large b} & {\dn\Large B} & {\dn\Large m} \\
pa & pha & ba & bha & ma \\*[0.25cm]

{\dn\Large y} & {\dn\Large r} & {\dn\Large l} & {\dn\Large v} & {\dn\Large f} \\
ya & ra & la & va & \'sa \\*[0.25cm]

{\dn\Large q} & {\dn\Large s} & {\dn\Large h} & {\dn\Large \322w} & {\dn\Large \3E2w} \\
\d{s}a & sa & ha & k\d{s}a & j\~na  \\*[0.25cm]
\end{tabular}

\end{center}



\newpage
\setcounter{page}{1}
\pagenumbering{roman}
\tableofcontents
\clearpage\thispagestyle{empty}\mbox{}\clearpage
\newpage
\pagenumbering{arabic}
\setcounter{page}{1}
\clearpage\pagestyle{plain}
\begin{center}
{\dn \399wFm\38Bwo-vAEm\7{t}lsFdAsjF EvrEct\\[0.1cm]
{\huge .. v\4rA`ys\306wdFpnF..}
}
\end{center}

{\dn\dnnum\large \textbf{dohA}{\rs --\re}rAm bAm EdEs jAnkF{\rs ,\re} lKn dAEhnF aor \flush .\pa
\hspace*{1cm} @yAn skl kSyAnmy{\rs ,\re} \7{s}rtz \7{t}lsF tor\tmark\flush.. \rn{1}..}

\trans{
\section{Invocation and description of God's Attributes}
\noindent\large 1. Lord \'Sr\={\i} R\=ama' has (his consort) J\=anak\={\i}j\={\i} on His left and (His brother) \'Sr\={\i} Lak\d{s}ma\d{n}a on his right---such a divine vision (visualisation) is ideal and all-benefitting (both temporal and spiritual). Tulas\={\i}d\=as\=a says that for him, this (such a vision) is an all-wish-fulfilling tree (the mythical Wish-tree of paradise).
}

{\dn\dnnum\large \7{t}lsF EmV\4 n moh tm{\rs ,\re} Ek{}e\1 koEV \7{g}n g\5Am \flush .\pa 
\3E3wdy kml \8{P}l\4 nhF{\qva}{\rs ,\re} Eb\7{n} rEb{\rs -\re}\7{k}l{\rs -\re}rEb rAm\tmark \flush .. \rn{2}.. }

\trans{\\*[0.1cm]
\large 2. Tulas\={\i}d\=as\=a says that despite imbibing numerous qualities, the darkness of ignorance cannot be removed, nor can the Lotus-heart blossom (i.e., the Divine spark, of Self-realisation, or Consciousness, or Pure self be brought to the fore) without R\=ama who is (like) the Sun (the Supreme fount of Divinity) of the Solar dynasty (here Tulas\={\i}d\=as\=a refers to the \textit{Sagu\d{n}a} incarnation of \textit{Nirgu\d{n}a} God, in the Solar dynasty as R\=ama.

\noindent\textbf{Note}---The emphasis is on both total surrender to God by the devotee as as well as His divine grace and mercy without which darkness cannot be removed.
}

{\dn\dnnum\large \7{s}nt lKt \7{\399w}Et nyn Eb\7{n}{\rs ,\re} rsnA Eb\7{n} rs l\?t \flush.\pa 
bAs nAEskA Eb\7{n} lh\4{\rs ,\re} prs\4 EbnA Enk\?t\tmark \flush .. \rn{3}.. }

\trans{\\*[0.1cm]
\large 3. Who hears without ears, sees without eyes, tastes without tongue, smells (breathes) without nose, and touches (feels) without skin (body) ? (Here, R\=ama has been ascribed \textit{Nirgu\d{n}a} or formless attributes, which are essentially the way God as been conceived in the \textit{Vedas} and other scriptures).
}

{\dn\dnnum\large \textbf{sorWA}{\rs --\re}aj a\392w\4t anAm{\rs ,\re} alK !p{\rs -\re}\7{g}n{\rs -\re}rEht jo \flush .\pa 
\hspace*{1.5cm}mAyA pEt so{}I rAm{\rs ,\re} dAs h\?\7{t} nr{\rs -\re}t\7{n}{\rs -\re}Dr\?{}u\tmark .. \rn{4}.. }

\trans{\\*[0.1cm]
\large 4. He who is unborn, without a second (non-dual), nameless, invisible (by the sense-organs of eyes), without form and qualities (attribute-less), and Lord of \textit{M\=ay\=a} (His creative potency) is the same `One' who has acquired a \textit{Sagu\d{n}a} form (incarnation of God) for the benefit of His devotees. (Note verses 3 \emph{\&} 4 bridge the gap between Formless and Formed variants of the Supreme being)
}

{\dn\dnnum\large \7{t}lsF yh t\7{n} K\?t h\4{\rs ,\re} mn bc km\0 EksAn \flush.\pa 
pAp \7{p}\306wy \392w\4 bFj h\4{\qva}{\rs ,\re} bv\4 so lv\4 EndAn\tmark \flush .. \rn{5}.. }

\trans{\\*[0.1cm]
\large 5. Tulas\={\i}d\=asa says that this body is like a (farmer's) field; the Mind (thought), Speech (talk), Action (activity) are the three farmers and good and bad deeds are two types of seeds. The harvest depends on the seeds sown (i.e., as you sow, so you reap) viz., this body is merely a producing medium. By wisely choosing the correct seeds of action using discrimination and intellect, the same body, in the hands of the three farmers can produce either good or bad results.
}

{\dn\dnnum\large \7{t}lsF yh t\7{n} tvA h\4{\rs ,\re} tpt sdA /\4tAp \flush .\pa
sAE\306wt ho{}I jb fAE\306wtpd{\rs ,\re} pAv\4 rAm \3FEwtAp\tmark \flush .. \rn{6}.. }

\trans{\\*[0.1cm]
\large 6. Tulas\={\i}d\=asa says that this body is like a fire-pan, which is always (at all-times) burning with the three fires of Spiritual, Temporal and Divine forces (i.e., is torn or regularly pulled apart between these three forces).\\
Peace and Relief (from this torment) is obtained only when one attains \textit{\'S\=anti-pada} (i.e., Supreme peace-giving feet of the Lord) which is possible only by \'Sr\={\i}-R\=ama's grace.\\
\textbf{Note}---The \textit{Trait\=apas} are---\textit{\=Adhy\=atmika}, \textit{\=Adhibhautika} and \textit{\=Adhidaivika}.
}

{\dn\dnnum\large \7{t}lsF b\?d \7{p}rAn{\rs -\re}mt{\rs ,\re} \8{p}rn fA-/ EvcAr \flush .\pa 
yh EbrAg{\rs -\re}s\306wdFpnF{\rs ,\re} aEKl `yAnko sAr\tmark \flush .. \rn{7}.. }

\trans{\\*[0.1cm]
\large 7. Tulas\={\i}d\=asa says that \textit{Vair\=agya Sand\={\i}pan\={\i}} contains the principles of the \textit{Vedas} and \textit{Pur\=a\d{n}as}, the views of all scriptures, and the essence of all Knowledge.
}

{\dn\dnnum\large \textbf{dohA}{\rs --\re}srl brn BAqA srl{\rs ,\re} srl aT\0my mAEn  \flush .\pa
\hspace*{1.1cm}\7{t}lsF srl\4 s\2tjn{\rs ,\re} tAEh prF pEhcAEn\tmark \flush .. \rn{8}..}

\trans{\section{Attributes of Saints}
\noindent\large 8. It (\textit{Vair\=agya Sand\={\i}pan\={\i}}) has simple words and language, its meaning is simple and straightforward.\\ Tulas\={\i}d\=asa says that saints of a pure and simple heart can easily understand it.
}

{\dn\dnnum\large
\textbf{cOpA{}I}{\rs --\re}\hspace*{0.2cm}aEt sFtl aEt hF \7{s}KdA{}I \\[0.1cm]
\hspace*{2.5cm}sm dm rAm Bjn aEDkA{}I   \flush .\pa
\hspace*{1.6cm}jX jFvn kO{\qva} kr\4 sc\?tA\\[0.1cm]
\hspace*{2.5cm}jg mh\1 Ebcrt h\4 eEh h\?tA\tmark \flush .. \rn{9}..}

\trans{\\*[0.1cm]
\large 9. Saints are of calm temperament and full and bliss. They have equitable and peaceful mind, self control (over the senses) and are especially known for devotion surrender to Lord R\=ama. They enlighten the ignorant ones, and for this reason (purpose) they wander in this world.
}

{\dn\dnnum\large \textbf{dohA}{\rs --\re}\7{t}lsF e\?s\? k\7{h}\1 k\8{h}\1{\rs ,\re} D\306wy DrEn vh s\2t  \flush .\pa
\hspace*{1.2cm}prkAj\? prmArTF{\rs ,\re} \3FEwFEt Ely\? Enbh\2t\tmark \flush .. \rn{10}..}

\trans{\\*[0.1cm]
\large 10. Tulas\={\i}d\=asa says such saints are rare, and that the land is blessed where they (such saints) are. They are ever engrossed in serving others and helping them in their spiritual quest, and persevere in this resolve with full devotion.
}

{\dn\dnnum\large kF \7{m}K pV dF\306wh\? rh\4{\qva}{\rs ,\re} jTA aT\0 BAq\2t  \flush .\pa
\7{t}lsF yA s\2sArm\?{\qva}{\rs ,\re} so EbcAr\7{j}t s\2t\tmark \flush .. \rn{11}..}

\trans{\\*[0.1cm]
\large 11. Tulas\={\i}d\=asa says that those saints who either keep quiet (do not speak unnecessarily, or speak only the truth, are te real, wise saints in this world.
}

{\dn\dnnum\large bol\4 bcn EbcAEr k\4{\rs ,\re} lF\306wh\?{\qva} s\2t \7{s}BAv  \flush .\pa
\7{t}lsF \7{d}K \7{d}b\0cn k\?{\rs ,\re} p\2T d\?t nEh\2 pA\1v\tmark \flush .. \rn{12}..}

\trans{\\*[0.1cm]
\large 12. Such saints think wisely before speaking and have acquired the temperament (mental bearing) of a true saint. Tulas\={\i}d\=asa says that they neither hurt anyone's feelings or sentiments, nor speak ill of others.
}

{\dn\dnnum\large s\7{/} n kA\8{h} kEr gn\4{\rs ,\re} Em/ gn\4 nEh\2 kAEh  \flush .\pa
\7{t}lsF yh mt s\2t ko{\rs ,\re} bol\4 smtA mAEh\tmark \flush .. \rn{13}..}

\trans{\\*[0.1cm]
\large 13. He (such saints) neither regards anyone as enemy nor a friend (i.e., al are equal in his eyes). Tulas\={\i}d\=asa says that it is the basic principle of a saint that he treats all equally with parity.
}

{\dn\dnnum\large
\textbf{cOpA{}I}{\rs --\re}\hspace*{0.2cm}aEt an\306wygEt i\2\qb{d}F jFtA\\
\hspace*{2.5cm}jAko hEr Eb\7{n} kt\7{h}\1 n cFtA  \flush .\pa
\hspace*{1.5cm}\9{m}g \9{t}\309wZA sm jg Ejy jAnF \\
\hspace*{2.5cm}\7{t}lsF tAEh s\2t pEhcAnF\tmark \flush .. \rn{14}..
}

\trans{\\*[0.1cm]
\large 14. He who has become one with the Supreme Being (i.e., who is totally devoted and surrendered to God and no one else), who has conquered his sense-organs, whose mind-intellect complex is concentrated on the Supreme Being (\'Sr\=i-Hari) and no one else, and who knows this world to be a mirage---Tulas\={\i}d\=asa says that such a man should be regarded as a saint.
}

{\dn\dnnum\large \textbf{dohA}{\rs --\re}ek Broso ek bl{\rs ,\re} ek aAs Eb-vAs  \flush .\pa
\hspace*{1.0cm} rAm!p -vAtF jld{\rs ,\re} cAtk \7{t}lsFdAs\tmark \flush .. \rn{15}..}

\trans{\\*[0.1cm]
\large 15. Tulas\={\i}d\=asa says that real saints are those who have only one reliance (on R\=ama), only one strength (R\=ama), only one source of expectation and faith, and for whom God's incarnation in the form of \'Sr\={\i}-R\=ama is like a dark rain-bearing cloud (of \textit{Sv\=at\={\i}-Nak\d{s}atra}) on which the bird called \textit{C\=ataka} has fixed it's gaze (i.e., are constantly concentrated on R\=ama's form).
}

{\dn\dnnum\large so jn jgt jhAj h\4{\rs ,\re} jAk\? rAg n doq  \flush .\pa
\7{t}lsF \9{t}\309wZA (yAEg k\4{\rs ,\re} gh\4 sFl s\2toq\tmark \flush .. \rn{16}..}

\trans{\\*[0.1cm]
\large 16. Tulas\={\i}d\=asa says that those who have neither attachment nor aversion with others, who have renounced greed and acquired (the virtues) of noble or righteous conduct and contentment, they are akin to ships (to ferry people from mundane to salvation).
}

{\dn\dnnum\large sFl ghEn sb kF shEn{\rs ,\re} khEn hFy \7{m}K rAm  \flush .\pa
\7{t}lsF rEh{}e eEh rhEn{\rs ,\re} s\2t jnn ko kAm\tmark \flush .. \rn{17}..}
	
\trans{\\*[0.1cm]
\large 17. According to Tulas\={\i}d\=asa, adhering to righteousness and noble conduct, tolerance towards all, always thinking and speaking of R\=ama (i.e. the Divine) and living (by example) such a life, are the (true) deeds of a saint.
}

{\dn\dnnum\large Enj s\2gF Enj sm krt{\rs ,\re} \7{d}rjn mn \7{d}K \8{d}n  \flush .\pa
mlyAcl h\4 s\2tjn{\rs ,\re} \7{t}lsF doq Eb\8{h}n\tmark \flush .. \rn{18}..}

\trans{\\*[0.1cm]
\large 18. Saints convert those who accompany them in their own likeness, but double (increase as much as twice) the anger (out of jealousy) of their adversaries (because saints hinder their nefarious designs). Tulas\={\i}d\=asa says that despite it (provocation from non-saints) they remain calm and without fault like the Malay\=acala (\textit{Cand\=ana} / sandal-wood).
}

{\dn\dnnum\large koml bAnF s\2t kF{\rs ,\re} \3FAwvt a\9{m}tmy aA{}i  \flush .\pa
\7{t}lsF tAEh kWor mn{\rs ,\re} \7{s}nt m\4n ho{}i jA{}i\tmark \flush .. \rn{19}..}

\trans{\\*[0.1cm]
\large 19. The words of a saint are so sweet and soft that they always produce a nectarean effect. Tulas\={\i}d\=asa says that even a hardened heart becomes malleable and soft (like wax) on hearing such words.
}

{\dn\dnnum\large a\7{n}Bv \7{s}K utpEt krt{\rs ,\re} ByB\5m Dr\4 uWA{}i  \flush .\pa
e\?sF bAnF s\2t kF{\rs ,\re} jo ur B\?d\4 aA{}i\tmark \flush .. \rn{20}..}

\trans{\\*[0.1cm]
\large 20. The word of a saint is such that it produces the bliss of realisation (of the Divine), puts aside fears and doubts, and penetrates (layers of ignorance of) the heart.
}

{\dn\dnnum\large sFtl bAnF s\2t kF{\rs ,\re} sEs\8{h} t\? a\7{n}mAn  \flush .\pa
\7{t}lsF koEV tpn hr\4{\rs ,\re} jo ko{}u DAr\4 kAn\tmark \flush .. \rn{21}..}

\trans{\\*[0.1cm]
\large 21. The cool, soothing words of a saint are far superior to (those qualities) of the moon. Tulas\={\i}d\=asa says that those who hear them, are rid of immense sufferings.
}

{\dn\dnnum\large 
\textbf{cOpA{}I}{\rs --\re}\hspace*{0.2cm}pAp tAp sb \8{s}l nsAv\4  \\[0.1cm]
\hspace*{2.5cm}moh a\2D rEb bcn bhAv\4 \flush .\pa
\hspace*{1.6cm}\7{t}lsF e\?s\? sd\7{g}n sA\8{D}  \\[0.1cm]
\hspace*{2.5cm}b\?d m@y \7{g}n EbEdt agA\8{D}\tmark \flush .. \rn{22}..
}

\trans{\\*[0.1cm]
\large 22. Saints destroy all (types of) sins, sufferings and agony; their words spread like the rays of the sun removing the darkness of ignorance. Tulas\={\i}d\=asa says that saints have such good qualities as are described (and made famous) by the \emph{Vedas}.
}

{\dn\dnnum\large \textbf{dohA}{\rs --\re}tn kEr mn kEr bcn kEr{\rs ,\re} kA\8{h} \8{d}Kt nAEh\2  \flush .\pa
\hspace*{1.2cm}\7{t}lsF e\?s\? s\2tjn{\rs ,\re} rAm!p jg mAEh\2\tmark \flush .. \rn{23}..}

\trans{\\*[0.1cm]
\large 23. Tulas\={\i}d\=asa says that those saints who never hurt anyone by their body (deeds), mind (thoughts) or word (speech) are the replica of Lord R\=ama in this world.\\*[0.1cm]
\textbf{Note:} Here the emphasis is on non-violence.
}


{\dn\dnnum\large \7{m}K dFKt pAtk hr\4{\rs ,\re} prst km\0 EblAEh\2  \flush .\pa
bcn \7{s}nt mn mohgt{\rs ,\re} \8{p}zb BAg EmlAEh\2\tmark \flush .. \rn{24}..}

\trans{\\*[0.1cm]
\large 24. Sins are vanquished on seeing their face, \emph{Karma} (accumulated result of all adulterated actions) is dissolved by their touch, the darkness (of ignorance) of mind is removed on hearing their (enlightened) words---such saints are found only due to good effects of past deeds i.e., destiny only (by luck only).
}

{\dn\dnnum\large aEt koml az Ebml zEc{\rs ,\re} mAns m\?{\qva} ml nAEh\2  \flush .\pa
\7{t}lsF rt mn ho{}i rh\4{\rs ,\re} apn\? sAEhb mAEh\2\tmark \flush .. \rn{25}..}

\trans{\\*[0.1cm]
\large 25. Saints are immensely soft (tender) and pure inclinations, and their heart has no scum. Tulas\={\i}d\=asa says that they are ever engrossed in the thought (contemplating on) of their Lord (R\=ama).
}

{\dn\dnnum\large jAk\? mn t\? uEW g{}I{\rs ,\re} EtlEtl \9{t}\309wZA cAEh  \flush .\pa
mnsA bAcA km\0nA{\rs ,\re} \7{t}lsF b\2dt tAEh\tmark \flush .. \rn{26}..}

\trans{\\*[0.1cm]
\large 26. Tulas\={\i}d\=asa pays homage from his heart, by his words and by his deeds to those (saints) from whose heart even the smallest speck of attachment and desire has gone away (i.e., removed).
}

{\dn\dnnum\large k\2cn kA\1cEh sm gn\4{\rs ,\re} kAEmEn kA\3A4w pqAn  \flush .\pa
\7{t}lsF e\?s\? s\2tjn{\rs ,\re} \9{p}LvF b\5\39Cw smAn\tmark \flush .. \rn{27}..}

\trans{\\*[0.1cm]
\large 27. Those saints who see no difference between gold and glass (treat valuable gold as worthles as a piece of glass) and those who regard women as (made of) wood or stone  (are not lustfully attracted to them) (i.e., have overcome greed and lust), Tulas\={\i}d\=asa says that such saints are akin to \emph{Brahma} (the Supreme Being) Himself upon the earth.
}


{\dn\dnnum\large 
\textbf{cOpA{}I}{\rs --\re}\hspace*{0.2cm}k\2cn ko \9{m}EtkA kEr mAnt\\[0.1cm]
\hspace*{2.5cm}kAEmEn kA\3A4w EslA pEhcAnt \flush .\pa
\hspace*{1.8cm}\7{t}lsF \8{B}El gyo rs eh  \\[0.1cm]
\hspace*{2.5cm}t\? jn \3FEwgV rAm kF d\?hA\tmark \flush .. \rn{28}..
}

\trans{\\*[0.1cm]
\large 28. Those (saints) who regard gold as worthless and mud (dust) and recognize (treat) women as a statue of wood or stone, Tulas\={\i}d\=asa says that they, who have forgotten this charm (of greed and lust), are just another form of Lord R\=ama.
}

{\dn\dnnum\large \textbf{dohA}{\rs --\re}aAEk\2cn i\2\qb{d}Fdmn{\rs ,\re} rmn rAm ik tAr  \flush .\pa
\hspace*{1.0cm} \7{t}lsF e\?s\? s\2t jn{\rs ,\re} Ebrl\? yA s\2sAr\tmark \flush .. \rn{29}..}

\trans{\\*[0.1cm]
\large 29. Tulas\={\i}d\=asa says that such saints, who have no worldly possessions, have fully controlled their sense-organs, and who single-mindedly concentrate on Lord R\=ama without diversion), are very rare in this world.
}

{\dn\dnnum\large ah\2bAd m\4{\qva} t\4{\qva} nhF{\qva}{\rs ,\re} \7{d}\3A3w s\2g nEh\2 ko{}i  \flush .\pa
\7{d}K t\? \7{d}K nEh\2 Upj\4{\rs ,\re} \7{s}K t\4{\qva} \7{s}K nEh\2 ho{}i\tmark \flush .. \rn{30}..}

\trans{\\*[0.1cm]
\large 30-31. Those who have neither ego nor distinguish between `mine' and `thine'; who have no evil company, who are neither affected by sad events or feel glad by happy events; who regard both gold and glass as worthless; for whom both foe and friend are equal (i.e., have none)---Tulas\={\i}d\=asa says that such people are called saints in this world.
}

{\dn\dnnum\large sm k\2cn kA\1c\4{\rs ,\re} Egnt s\7{/} Em/ sm do{}i  \flush .\pa
\7{t}lsF yA s\2sArm\?{\qva}{\rs ,\re} kAt s\2t jn so{}I \flush .. \rn{31}..}

{\dn\dnnum\large Ebrl\? Ebrl\? pA{}i{}e{\rs ,\re} mAyA (yAgF s\2t  \flush .\pa
\7{t}lsF kAmF \7{k}EVl kEl{\rs ,\re} k\?kF k\?k an\2t\tmark \flush .. \rn{32}..}

\trans{\\*[0.1cm]
\large 32. Tulas\={\i}d\=asa says that in \emph{Kaliyuga} saints who have completely renounced \emph{M\=aya} are extremely rare, but there is no dearth of those people (non-saints) who are sweet-talking and selfish (back-stabbers) like the peacock and peahen (who gobble up snakes at the first opportunity despite their sweet and beautiful countence).
}

{\dn\dnnum\large m\4{\qva} t\2 m\?\3D4wo moh tm{\rs ,\re} u`yo aAtmA BA\7{n}  \flush .\pa
s\2t rAj so jAEny\?{\rs ,\re} \7{t}lsF yA sEhdA\7{n}\tmark \flush .. \rn{33}..}

\trans{\\*[0.1cm]
\large 33. Those (saints) who have conquered ego, from whom the darkness of ignorance  of `Mine \emph{\&} Thine' factor has been eliminated, and in whose heart the `Sun of Knowledge of Self-realisation' has risen---Tulas\={\i}d\=asa says that such (saints) should be recognised as King among saints. This is the characteristic mark of a saint.
}
{\dn\dnnum\large \textbf{sorWA}{\rs --\re}ko brn\4 \7{m}K ek{\rs ,\re} \7{t}lsF mEhmA s\2t kF  \flush .\pa
\hspace*{1.0cm} Ej\306wh k\? Ebml Ebb\?k{\rs ,\re} s\?s mh\?s n kEh skt\tmark .. \rn{34}..}

\trans{\\*[0.1cm]
\section{The Glory of Saints}
\large 34. Tulas\={\i}d\=asa wonders who can describe the glory of sing the praises of a saint by a single mouth. Even the thousand hooded mythical serpent \emph{(\'Se\'sa-n\=aga)} and five mouthed Mahe\'svara who are stepped in pure wisdom are unable to do so.
}

{\dn\dnnum\large \textbf{dohA}{\rs --\re}mEh p/F kEr Es\2\7{D} mEs{\rs ,\re} tz l\?KnF bnA{}i  \flush .\pa
\hspace*{1.0cm} \7{t}lsF gnpt so{\qva} tdEp{\rs ,\re} mEhmA ElKF n jA{}i\tmark \flush .. \rn{35}..}

\trans{\\*[0.1cm]
\large 35. Tulas\={\i}d\=asa says that the importance and praises of a saint cannot be written (described or narrated) even by Lord Ga\d{n}e\'sa Himself on a paper as large as the earth, using water of the oceans as ink and mythical \emph{Kalpa} tree as the pen (i.e., they are beyong description).
}

{\dn\dnnum\large D\306wy D\306wy mAtA EptA{\rs ,\re} D\306wy \7{p}/ br so{}i  \flush .\pa
\7{t}lsF jo rAmEh Bj\?{\rs ,\re} j\4s\?\7{h}\1 k\4s\?\7{h}\1 ho{}i\tmark \flush .. \rn{36}..}

\trans{\\*[0.1cm]
\large 36. Tulas\={\i}d\=asa says that those parents are blessed and their son is blessed and best, who devotes and surrenders himself to R\=ama in anyway whatever.
}

{\dn\dnnum\large \7{t}lsF jAk\? bdn t\?{\rs ,\re} DoK\?\7{h}\1 Enkst rAm  \flush .\pa
tAk\? pg kF pgtrF{\rs ,\re} m\?r\? tn ko cAm\tmark \flush .. \rn{37}..}

\trans{\\*[0.1cm]
\large 37. Tulas\={\i}d\=asa says that he will feel honoured if those who utter the (Holy) name of R\=ama even unwillingly and inadvertently, have their footwear made from his hide (skin).
}

{\dn\dnnum\large \7{t}lsF Bgt \7{s}pc BlO{\rs ,\re} Bj\4 r\4n Edn rAm  \flush .\pa
U\1co \7{k}l k\?Eh kAmko{\rs ,\re} jhA\1 n hErko nAm\tmark \flush .. \rn{38}..}

\trans{\\*[0.1cm]
\large 38. Tulas\={\i}d\=asa says that even an outcast \emph{(C\=a\d{n}\d{d}\=ala)} who chants the (Holy) name of R\=ama day and night, is far better than an upper-caste (household) where is no name (constant remembrance) of R\=ama.
}

{\dn\dnnum\large aEt U\1c\? \8{B}DrEn pr{\rs ,\re} \7{B}jgn k\? a-TAn  \flush .\pa
\7{t}lsF aEt nFc\? \7{s}Kd{\rs ,\re} UK a\3E0w az pAn\tmark \flush .. \rn{39}..}

\trans{\\*[0.1cm]
\large 39. Tulas\={\i}d\=asa says that on a high mountain reside poisonous snakes and serpents (which are dangerous), while on low plains grow sugarcane, cereals and betel leaves.\\(Meaning that, in devotion-less high-castes, there are negative traits such as pride, haughtiness, lust, anger etc., while in devoted low-castes there are positive traits such as surrender to God, peace, happiness etc. Hence, they are better.
}

{\dn\dnnum\large
\textbf{cOpA{}I}{\rs --\re}\hspace*{0.2cm}aEt an\306wy jo hEr ko dAsA\\[0.1cm]
\hspace*{2.5cm}rV\4 nAm EnEsEdn \3FEwEt -vAsA \flush .\pa
\hspace*{1.8cm}\7{t}lsF t\?Eh smAn nEh\2 ko{}I \\[0.1cm]
\hspace*{2.5cm}hm nFk\?{\qva} d\?KA sb ko{}I\tmark \flush .. \rn{40}..
}

\trans{\\*[0.1cm]
\large 40. He who is totally surrendered to Lord \emph{Hari} and chants His name day and night with each breath, Tulas\={\i}d\=asa says that after observing all people well, he has come to the conclusion that such a man has no equal.
}

{\dn\dnnum\large 
\hspace*{1.8cm}jdEp sA\7{D} sbhF EbED hFnA\\[0.1cm]
\hspace*{2.5cm}t\38DwEp smtA k\? n \7{k}lFnA\ \flush .\pa
\hspace*{1.8cm}yh Edn r\4n nAm u\3CEwr\4\\[0.1cm]
\hspace*{2.5cm}vh Ent mAn aEgEn mh\1 jr\4\tmark \flush .. \rn{41}..}

\trans{\\*[0.1cm]
\large 41. A pure hearted hermit (even of low-caste) though he possesses nothing and is utterly humble, is superior in comparison to a upper-caste born, because the former chants the Lord's name day and night while the latter burns daily in the fire of false-pride (of upper-caste birth).
}

{\dn\dnnum\large  \textbf{dohA}{\rs --\re}dAs rtA ek nAm so{\qva}{\rs ,\re} uBy lok \7{s}K (yAEg  \flush .\pa
\hspace*{1.1cm}\7{t}lsF \306wyAro \39Fw\4 rh\4{\rs ,\re} dh\4 n \7{d}K kF aAEg\tmark \flush .. \rn{42}..}

\trans{\\*[0.1cm]
\large 42. The Lord's servant (one who has surrendered himself before the Lord) loves only the name of the Lord, leaving aside the pleasures of this world and heaven (after death). Tulas\={\i}d\=asa says that such a person lives a unique life of detachment, and hence does not suffer (burn) from the fire of sorrow.
}
{\dn\dnnum\large r\4En ko \8{B}qn i\2\7{d} h\4{\rs ,\re} Edvs ko \8{B}qn BA\7{n}  \flush .\pa
dAs ko \8{B}qn BE\3C4w h\4{\rs ,\re} BE\3C4w ko \8{B}qn `yA\7{n} \flush .. \rn{43}..}


{\dn\dnnum\large `yAn ko \8{B}qn @yAn h\4{\rs ,\re} @yAn ko \8{B}qn (yAg  \flush .\pa
(yAg ko \8{B}qn fA\2Etpd{\rs ,\re} \7{t}lsF aml adAg\tmark \flush .. \rn{44}..}


\trans{\\*[0.1cm]
\section{Description of Peace}
\large 43-44. The moon is the jewel of the night; the sun adorns the day; a \emph{Bhakta's} (one who has surrendered himself to the Lord) ornament is his devotion; knowledge (of \emph{Brahman}) is the ornament of devotion; concentration (meditation) on the Pure-Self is the ultimate aim of concentration; renunciation is the (result) of ornament of concentration; and---according to Tulas\={\i}d\=asa, realisation of Eternal Bliss (of the Eternal feet of the Lord) is the pinnacle of renunciation, which (itself) is pure and without blemish.
}

{\dn\dnnum\large 
\textbf{cOpA{}I}{\rs --\re}\hspace*{0.2cm}aml adAg fA\2Etpd sArA \\  
\hspace*{2.5cm}skl kl\?s n krt \3FEwhArA \flush .\pa
\hspace*{1.8cm}\7{t}lsF ur DAr\4 jo ko{}I \\[0.1cm]
\hspace*{2.5cm}rh\4 an\2d Es\2\7{D} mh\1 so{}I\tmark \flush .. \rn{45}..
}

\trans{\\*[0.1cm]
\large 45. This pure and without blemish Eternal Blissful Feet of the Lord is the essence of all (knowledge), on realisation (attachment) of which no sorrows emanating from ignorance can make any attack (i.e., cannot do any harm). Tulas\={\i}d\=asa says that those who keep it (Eternal Blissful Feet of the Lord) in their heart are at all times immersed in everlasting peace and happiness.
}

{\dn\dnnum\large 
\hspace*{1.8cm}EbEbD pAp s\2Bv jo tApA\\
\hspace*{2.5cm}EmVEh\2 doq \7{d}K \7{d}sh klApA  \flush .\pa
\hspace*{1.8cm}prm sA\2Et \7{s}K rh\4 smA{}I  \\
\hspace*{2.5cm}th\1 utpAt n b\?D\4 aA{}I\tmark \flush .. \rn{46}..}

\trans{\\*[0.1cm]
\large 46. The sufferings which are caused by various sins, and faults and unbearable sorrows are vanquished. He (who has realised that \emph{\'S\=anti-Pada}) enters that blissful state of mind which no violence can disturb.
}

{\dn\dnnum\large
\hspace*{1.8cm}\7{t}lsF e\?s\? sFtl s\2tA  \\
\hspace*{2.5cm}sdA rh\4 eEh BA\1Et ek\2tA \flush .\pa
\hspace*{1.8cm}khA kr\4 Kl log \7{B}j\2gA \\
\hspace*{2.5cm}kF\306w\39DwO grlsFl jo a\2gA\tmark \flush .. \rn{47}..}

\trans{\\*[0.1cm]
\large 47. Tulas\={\i}d\=asa says that such cool (peaceful and placid) saints live a solitary life (i.e., live alone immersed in the thought of \emph{\'Santi-Pada} and bereft of attachment to the world). What harms can scoundrels, who have converted themselves like a poisonous snake (i.e., who always harm those who come to in contact with them), do them (saints who have attained \emph{\'Santi-Pada}).
}

{\dn\dnnum\large \textbf{dohA}{\rs --\re}aEt sFtl aEthF aml{\rs ,\re} skl kAmnA hFn  \flush .\pa
\hspace*{1.2cm}\7{t}lsF tAEh atFt gEn{\rs ,\re} \9{b}E\381w sA\2Et lylFn\tmark \flush .. \rn{48}..}

\trans{\\*[0.1cm]
\large 48. Those who are of very cool (placid) temperament (i.e., give solace to others) are immensely pure, are without any desires whatsoever, and whose thoughts are immersed in Eternal Bliss---Tulas\={\i}d\=asa says that such saints should be considered transcending the world and worldliness.
}


{\dn\dnnum\large 
\textbf{cOpA{}I}{\rs --\re}\hspace*{0.2cm}jo ko{}i kop Br\? \7{m}K b\4nA \\[0.1cm]
\hspace*{2.5cm}s\306w\7{m}K ht\4 EgrAsr p\4nA \flush .\pa
\hspace*{1.8cm}\7{t}lsF t{}U l\?s Ers nAEh\2 \\ 
\hspace*{2.5cm}so sFtl kEh{}e jg mAhF{\qva}\tmark \flush .. \rn{49}..}

\trans{\\*[0.1cm]
\large 49. Tulas\={\i}d\=asa says that even when someone speaks acrimoniously in anger and showers angry words sharp as arrows, those who are not the least ruffled by them (such words) are said to be cool (placid) saints in this world.
}

{\dn\dnnum\large \textbf{dohA}{\rs --\re}sAt dFp nv K\2X lO{\rs ,\re} tFEn lok jg mAEh\2  \flush .\pa
\hspace*{1.1cm}\7{t}lsF sA\2Et smAn \7{s}K{\rs ,\re} apr \8{d}sro nAhF{\qva}\tmark \flush .. \rn{50}..}

\trans{\\*[0.1cm]
\large 50. Tulas\={\i}d\=asa says that there is no greater happiness as compared to peace of mind and heart anywhere in the seven continents, nine \emph{Kha\d{n}\d{d}as} (parts or segments of the earth), and nor even in the three \emph{Lokas} (i.e., Heaven or \emph{Svarga-loka}, Earth or \emph{Bh\=u-loka} and subterranean or \emph{P\=at\=ala-loka}).
}


{\dn\dnnum\large
\textbf{cOpA{}I}{\rs --\re}\hspace*{0.2cm}jhA\1 sA\2Et st\7{g}z kF d{}I  \\[0.1cm]
\hspace*{2.5cm} thA\1 \387woD kF jr jEr g{}I \flush .\pa
\hspace*{1.8cm}skl kAm bAsnA EblAnF \\[0.1cm]
\hspace*{2.5cm}\7{t}lsF bh\4 sA\2Et sEhdAnF\tmark \flush .. \rn{51}..
}

\trans{\\*[0.1cm]
\large 51. As soon as peace is given by \emph{Sat-guru} (enlightened and truthful teacher) is received (by the disciple), his root of anger is burnt (i.e., destroyed), and all his desires and lust and vanished. Tulas\={\i}d\=asa says that this is the test of true peace.
}

{\dn\dnnum\large 
\hspace*{1.8cm}\7{t}lsF \7{s}Kd sA\2Et ko sAgr \\
\hspace*{2.5cm}s\2tn gAyo krn ujAgr \flush .\pa
\hspace*{1.8cm}tAm\?{\qva} tn mn rh\4 smo{}I  \\
\hspace*{2.5cm}ah\2 aEgEn nEh\2 dAh\4{\qva} ko{}I\tmark \flush .. \rn{52}..
}

\trans{\\*[0.1cm]
\large 52. Tulas\={\i}d\=asa says that the fire of false-pride and ego can never burn someone who is immersed body and mind in to what has been described by saints as dispenser of happiness, ocean of peace and source of light of enlightenment (i.e.,  one who has realised  \emph{\'S\=anti-pada} as described in these verse, or knowledge of The self of \emph{Brahma}.
}

{\dn\dnnum\large\textbf{dohA}{\rs --\re}ah\2kAr kF aEgEn m\?{\qva}{\rs ,\re} dht skl s\2sAr  \flush .\pa
\hspace*{1.2cm}\7{t}lsF bA\1c\4 s\2tjn{\rs ,\re} k\?vl sA\2Et aDAr\tmark \flush .. \rn{53}..}

\trans{\\*[0.1cm]
\large 53. The whole world is burning in the fire of false-pride (or egoism). Tulas\={\i}d\=asa says that only saints are saved from it for they take shelter in inner peace of \emph{\'S\=anti-pada}.
}

{\dn\dnnum\large mhA sA\2Et jl prEs k\4{\rs ,\re} sA\2t B{}e jn jo{}i  \flush .\pa
ah\2 aEgEn t\? nEh\2 dh\4{\qva}{\rs ,\re} koEV kr\4 jo ko{}i\tmark \flush .. \rn{54}..}

\trans{\\*[0.1cm]
\large 54. Those saints who have attained peace on contact with the water of  (Eternal or) Great Peace, do not burn from the fire of false-pride (egoism) no matter how hard anyone tries (to entice them).
}

{\dn\dnnum\large t\?j hot tn trEn ko{\rs ,\re} acrj mAnt lo{}i  \flush .\pa
\7{t}lsF jo pAnF ByA{\rs ,\re} b\7{h}Er n pAvk ho{}i\tmark \flush .. \rn{55}..}

\trans{\\*[0.1cm]
\large 55. The bright halo (glow of righteousness) effusing from the body of a saint who has gone beyond egoism is like that of the sun; people are surprised at it. Tulas\={\i}d\=asa says that those who have become (cool as) water cannot become (hot as) fire again.
}

{\dn\dnnum\large j\38DwpF sFtl sm \7{s}Kd{\rs ,\re} jgm\?{\qva} jFvn \3FEwAn  \flush .\pa
tdEp sA\2Et jl jEn gnO{\rs ,\re} pAvk t\?l \3FEwmAn\tmark \flush .. \rn{56}..}

\trans{\\*[0.1cm]
\large 56.  Though it \emph{(\'S\=anti-pada)} is cool, equitable and blissful, and the very life element of the (saints of the) world, yet it should not be mistaken for being (cool as) water because it as potent as fire.
}

{\dn\dnnum\large 
\textbf{cOpA{}I}{\rs --\re}\hspace*{0.2cm}jr\4 br\4 az KFEJ EKJAv\4 \\[0.1cm]
\hspace*{2.5cm}rAg \392w\?q mh\1 jnm g\1vAv\4 \flush .\pa
\hspace*{1.8cm}spn\?\7{h}\1 sA\2Et nEh un d\?hF \\[0.1cm]
\hspace*{2.5cm}\7{t}lsF jhA\1jhA\1 b\5t ehF\tmark \flush .. \rn{57}..
}

\trans{\\*[0.1cm]
\large 57. Those who are always burning (in the fire of ego and desire), are angry themselves and make others angry (by their behaviour), and spend their lives in attachment and repulsion---Tulas\={\i}d\=asa says that wherever there is such a temperment, there can never be true peace in the wildest of dreams.
}

{\dn\dnnum\large \textbf{dohA}{\rs --\re}so{}i p\2EXt so{}i pArKF{\rs ,\re} so{}I s\2t \7{s}jAn  \flush .\pa
\hspace*{1.2cm}so{}I \8{s}r sc\?t so{\rs ,\re} so{}I \7{s}BV \3FEwmAn\tmark \flush .. \rn{58}..}

\trans{\\*[0.1cm]
\large 58-59. Tulas\={\i}d\=asa says that from whose mind and thought both likes and dislikes have been banished, they are the true experts (in the essence of scriptures), true elevators (of qualities) clever, saints, brave, alert, proved warriors, truly enlightened and wise ones, benevolent and real meditators on the True Self \emph{(Brahman)}.
}

{\dn\dnnum\large so{}i `yAnF soi \7{g}nF jn{\rs ,\re} so{}I dAtA @yAEn  \flush .\pa
\7{t}lsF jAk\? Ect B{}I{\rs ,\re} rAg \392w\?q kF hAEn \flush .. \rn{59}..}



{\dn\dnnum\large 
\textbf{cOpA{}I}{\rs --\re}\hspace*{0.2cm}rAg \392w\?q kF aEgEn \7{b}JAnF  \\
\hspace*{2.5cm}kAm \387woD bAsnA nsAnF \flush .\pa
\hspace*{1.8cm}\7{t}lsF jbEh sA\2Et \9{g}h aA{}I\\
\hspace*{2.5cm}tb urhF{\qva} ur EPrF dohA{}I\tmark \flush .. \rn{60}..}

\trans{\\*[0.1cm]
\large 60. When the fire of (worldly) likes and dislikes has been doused; when desire, anger and lust have been vanished;  and when peace has entered the inner-self (heart-mind-intellect)---Tulas\={\i}d\=asa says that then only it is possible to invoke the kingdom of God to rule over the (inner) self.
}

{\dn\dnnum\large \textbf{dohA}{\rs --\re}EPrF dohA{}I rAm kF{\rs ,\re} g\? kAmAEdk BAEj  \flush .\pa
\hspace*{1.2cm}\7{t}lsF >yo{\qva} rEb k\?{\qva} udy{\rs ,\re} \7{t}rt jAt tm lAEj\tmark .. \rn{61}..}

\trans{\\*[0.1cm]
\large 61. Tulas\={\i}d\=asa says that as soon as the kingdom of God is invoked in the inner-self (i.e., the inner-self is lightened up with the brightness of God's exuberance), the desire etc., ran away from the heart in a similar way as darkness shys away (goes away) when the sun rises.
}

{\dn\dnnum\large yh EbrAg s\2dFpnF{\rs ,\re} \7{s}jn \7{s}Ect \7{s}En l\?\7{h}  \flush .\pa
a\7{n}Ect bcn EbcAEr k\?{\rs ,\re} js \7{s}DAEr ts d\?\7{h}\tmark \flush .. \rn{62}..}

\trans{\\*[0.1cm]
\large 62. Gentlemen! (Earnest devotees!) Please listen and pay attention to this \emph{Vair\=agya Sand\={\i}pan\={\i}} carefully and with full concentration; if any wrong (or incorrect) word / phrase (idea / thought) is found anywherem then please check / correct it after giving due (wise) thought.
}
\vspace*{\fill}
\begin{center}
{\dn\Large sm\3D8w\396wAy\2 g\5\306wT,\tmark}\\
{\dn\dnpen .. \399wFsFtArAmc\306w\qb{d}Ap\0Zm-\7{t} ..}
\end{center}

\trans{\\*[0.1cm]
\begin{center}
{\itshape Thus ends the \emph{Vair\=agya Sand\={\i}pan\={\i}} composed by\\
\'Sr\={\i} Gosv\=am\={\i} Tulas\={\i}d\=asa.}
\end{center}
}
\vspace*{\fill}
\newpage
\thispagestyle{empty}
\section*{Colophon}

This document was typeset using the \LaTeX\  software along with the `Geometry', `Devanagari' and `Titlesec' packages.

The original verses are typeset in `large' point size in the `Devanagari' font which comes with the \emph{Devanagari} package, while the English translation is in `normalsize' point size in `Computer Modern' font, which comes with \LaTeX\ software.

The original verses are largely based on the source from the `Sanskrit Documents' website\textsuperscript{1}\footnote[1]{\textsuperscript{1}http://www.sanskritdocuments.org} and were modified for better readability. Specifically `commas' are inserted in \emph{Dohas} and \emph{Sor\d{t}has} where a brief pause is given during recitation; apart from a few corrections in the text. The translation is large based on that which appeared in `Kalyana-Kalpataru' issues February through May 2004 penned by Ajai Kumar Chhawchharia.

This document is most humbly submitted with devotion unto Lord \'Sr\={\i}-R\=ama, whose Lotus feet twain are alone the means to cross this unfathomable ocean of transmigratory existence and who alone shines forth as the One Truth, The Supreme \emph{Brahman}\textsuperscript{2}\footnote[2]{\textsuperscript{2}viz. \emph{\'Sr\={\i}-R\=amacarita-M\=anasa} - I.6}.

\begin{center}
{\dn\dnpen\itshape .. \399wFsFtArAmc\306w\qb{d}Ap\0Zm-\7{t} ..}
\end{center}
\end{document}